% File: apm-33-neutrinos-weak-sector-qmu.tex
% !TEX program = pdflatex
\documentclass[coverpage]{qadi-article}

% ========== Optional theorem envs / tables ==========
\usepackage{amsthm}
\usepackage{booktabs}

% ========== Metadata ==========
\title{Neutrinos and the Weak Sector in the Aether Physics Model (QMU):\\
Chronovibrational Mixing, Mass Ratios, and Oscillations}
\author{\QADIAuthor{David W. Thomson III}{0000-0002-5830-5427}}
\date{\QADIDate}

% ========== Theorem-like envs ==========
\newtheorem{theorem}{Theorem}
\newtheorem{lemma}{Lemma}
\newtheorem{proposition}{Proposition}

% ========== Math operators / helpers ==========
\DeclareMathOperator{\Arg}{Arg} % principal complex argument

\begin{document}
\QADIMakeTitle
\QADIKeywords{Aether Physics Model; QMU; neutrino mass ratios; oscillation phases; chronovibration; holonomy; Born--Infeld caps}
% ==============================================================
\begin{abstract}
We formulate a QMU-native weak sector within the Aether Physics Model (APM). Neutrino mass and mixing emerge from chronovibrational couplings to a dual Aether gauge sector, with loxodromic holonomy acting as a geometric surrogate for the PMNS map \cite{Maki1962,GribovPontecorvo1969}. Intrinsic curvature and torsion of chronovibrational flow, obtained from dual-field invariants under a Born--Infeld cap, determine the holonomy \(U(H)\) and hence the spectrum of mass \emph{ratios} and oscillation phases. A minimal ansatz yields the three mixing angles and phase differences directly from QMU invariants, mapped to PDG observables without SI translation \cite{SuperK1998,SNO2002,KamLAND2003,DayaBay2012,PDG2024Neutrino}. Born--Infeld caps enforce finiteness and bound effective Majorana-like phases, giving falsifiable inequalities on CP-odd terms \cite{BI1934,Dirac1960BI,FradkinTseytlin1985}.
\end{abstract}

% ==============================================================
\section*{Nomenclature (brief)}
\textbf{Ledger-first/QMU.} All quantities arise from the base set \(\BaseSet\) obeying the ledger identity
\begin{equation}
\LedgerOne.
\label{eq:ledger}
\end{equation}
\textbf{Chronovibration.} Intrinsic time-flip dynamics of the Aether lattice at frequency \(\Fq\).  
\textbf{Loxodrome.} Worldline of constant pitch relative to chronovibrational flow.  
\textbf{\(\rson\).} Chronovibration-parity operator acting on a two-state internal space.

% ==============================================================
\section{QMU weak blocks and chronovibrational couplings}
We posit a dual U(1) Aether pair \((\mathcal A_\mu,\widetilde{\mathcal A}_\mu)\) with curvatures \((\mathcal F_{\mu\nu},\widetilde{\mathcal F}_{\mu\nu})\).  
Chiral projectors are
\begin{equation}
P_\pm=\tfrac12(\mathbb1\pm\rson),\qquad P_\pm^2=P_\pm,\quad P_+P_-=0,
\label{eq:projectors}
\end{equation}
with \(\rson^\dagger=\rson,\ \rson^2=\mathbb1\).
\emph{Remark.} Taking \(\mathbb1\) as identity ensures \(P_\pm^2=P_\pm\).
In the minimal realization, \(\rson=\sigma_3\) acts on the chronovibration doublet \(\Psi=(\psi_\uparrow,\psi_\downarrow)\).

\paragraph{Minimal weak coupling.}
\[
\mathcal L_{\text{wk}}
=\Psi^\dagger[i\partial\!\!\!/-\mathcal W\!\!\!/]\Psi,\qquad
\mathcal W_\mu=g_+\mathcal A_\mu P_+ + g_- \widetilde{\mathcal A}_\mu P_-,
\]
with ledger-dimensionless couplings \(g_\pm\).
This geometrizes the PMNS mechanism \cite{Maki1962,GribovPontecorvo1969} without Yukawa insertions.

% ==============================================================
\section{From dual fields to holonomy: dynamics, not a black box}
Field invariants under the Born--Infeld cap:
\begin{equation}
\mathcal I=\tfrac12\mathrm{tr}(\mathcal F^2-\widetilde{\mathcal F}^2),\qquad
\mathcal J=\mathrm{tr}(\mathcal F\widetilde{\mathcal F}),\qquad
\frac{\mathcal I^2+\mathcal J^2}{(\eemaxsq)^4}\le1.
\label{eq:BI}
\end{equation}
The Frenet–Serret curvature and torsion are taken as
\begin{equation}
\kappa=\kappa_0\frac{\sqrt{\mathcal I}}{\eemaxsq},\qquad
\tau=\tau_0\frac{\mathcal J}{\eemaxsq\sqrt{\mathcal I}},
\label{eq:kappatau}
\end{equation}
dimensionless and cap-bounded.

Holonomy along a loxodrome:
\[
H[\gamma]=\mathcal P\exp\!\Big(\int_\gamma\bm\Omega\Big),\qquad
\bm\Omega=\kappa\,\bm n\,ds+i\tau\,\bm t\,ds,
\]
yielding unitary \(U(H)\) analogous to PMNS.

% ==============================================================
\section{Naturalness of the minimal ansatz and robustness}
\paragraph{Why \(\kappa\!\propto\!\sqrt{\mathcal I}\), \(\tau\!\propto\!\mathcal J/\sqrt{\mathcal I}\).}
Parity (even/odd duality), cap saturation, and regularity at small fields single out this lowest-order map. Other monotone choices exist but preserve equivalent behavior.

\paragraph{Sensitivity.}
For \(u=\sqrt{\mathcal I}/\eemaxsq,\ v=\mathcal J/(\eemaxsq\sqrt{\mathcal I})\), small perturbations shift \(\Theta=S\sqrt{\kappa^2+\tau^2}\) by
\[
\frac{\delta\Theta}{\Theta}=
\frac{\kappa^2}{\kappa^2+\tau^2}\frac{\delta u}{u}+
\frac{\tau^2}{\kappa^2+\tau^2}\frac{\delta v}{v},
\]
demonstrating smooth response; no fine tuning.

% ==============================================================
\section{Mass generation and ratios}
From \eqref{eq:ledger},
\[
X=\frac{\eemaxsq}{\eesq},\qquad
Y=\frac{\Fq\lC}{\sqrt{\Au\curlapm}}=1.
\]
Ledger-closed inertial response scales with \(X\); holonomy injects geometric eigenvalues \(\lambda_i(H)\):
\begin{equation}
\frac{m_{\nu_i}}{\me}=\alpha X^{\beta}\lambda_i(H)^{\gamma},
\quad
\frac{m_{\nu_i}}{m_{\nu_j}}=\!\left(\frac{\lambda_i(H)}{\lambda_j(H)}\right)^{\!\gamma}.
\label{eq:ratio}
\end{equation}

% ==============================================================
\section{Oscillation phases and baselines}
\[
\Phi_{ij}[\gamma]=\!\int_\gamma\!\Delta\mu_{ij}\frac{ds}{\lC}+\Delta\theta_{ij}(H),
\quad
\Delta\mu_{ij}=\frac{m_{\nu_i}^2-m_{\nu_j}^2}{\me^2}.
\]
Transition probability:
\[
\mathcal P_{\alpha\to\beta}=\delta_{\alpha\beta}
-4\!\sum_{i>j}\!\Re[\cdots]\sin^2\!\frac{\Phi_{ij}}{2}
+2\!\sum_{i>j}\!\Im[\cdots]\sin\Phi_{ij}.
\]

Parameterization:
\[
U(H)=R_{23}(\vartheta_{23})R_{13}(\vartheta_{13},\delta_{\rm geo})R_{12}(\vartheta_{12}),
\]
with \(\delta_{\rm geo}\) extracted in the PDG way (e.g., from the Jarlskog invariant \(J\)); note \(\det U(H)=1\Rightarrow \Arg\det U=0\).

% ==============================================================
\section{Constraints, predictions, and PDG mapping}
Born–Infeld phase caps:
\[
|\Xi_k|=\bigl|\Arg\lambda_k(H)\bigr|\le\Xi_{\max}(X),\qquad X=\eemaxsq/\eesq.
\]
Double-ratio and baseline-rescaling tests depend only on geometry \(U(H)\).

% ==============================================================
\section*{Discussion}
We have demonstrated that weak phenomena co-arise within the Aether Physics Model as an emergent property of chronovibrational coupling to a dual Aether sector.  
The ledger-first construction yields a predictive holonomy determined by field invariants under a single Born--Infeld cap.  
Neutrino oscillation data are therefore interpreted as direct probes of the loxodromic geometry of the Aether rather than as mass differences expressed in SI units; quantitative comparisons to global fits remain straightforward \cite{PDG2024Neutrino}.

Although this study has concentrated on the weak sector, the underlying principles of ledger closure and chronovibrational dynamics are general.  
The same formalism can extend to other interaction domains and to large-scale Aether structures, where curvature–resonance reciprocity may play a similar role.  
These broader implications—including possible links to dark-sector and gravitational phenomena—will be developed in subsequent papers.

% ==============================================================
\section*{Methods (ledger-first)}
All relations derive from \(\BaseSet\) and \eqref{eq:ledger}.  
Loxodromic geometry uses the Frenet–Serret frame; holonomy follows the matrix exponential form below.  
Probabilities use Eq.~\eqref{eq:ratio} with \(U(H)\) from the SU(3) embedding.

\paragraph{Generator basis and loxodromic tilt.}
We realize plane mixings with SU(3) generators as
\(T_{12}^{(a)}=\lambda_2/2,\ T_{12}^{(s)}=\lambda_1/2,\ 
T_{23}^{(a)}=\lambda_7/2,\ T_{23}^{(s)}=\lambda_6/2,\ 
T_{13}^{(a)}=\lambda_5/2,\ T_{13}^{(s)}=\lambda_4/2.\)
Torsion–curvature coupling introduces a tilt
\[
\widetilde T_{ij}(\phi)=\cos\phi\,T_{ij}^{(a)}+\sin\phi\,T_{ij}^{(s)}.
\]
Then
\[
U(H)=\exp[-i\,\Theta_{\rm eff}G],\quad
G=n_{12}\widetilde T_{12}(\phi)+n_{23}\widetilde T_{23}(\phi)+n_{13}\widetilde T_{13}(\phi),
\]
with normalized axis \(\bm n=(n_{12},n_{23},n_{13})\).

% ==============================================================
\appendix
\section*{Appendix A: Optional translation (out-of-ledger)}
For comparison with PDG data,
\[
\int_\gamma\!\frac{ds}{\lC}\leftrightarrow\frac{L}{\lC},\qquad
\Delta\mu_{ij}\leftrightarrow\frac{\Delta m_{ij}^2}{\me^2}.
\]
Absolute SI conversions belong here; the main text remains QMU-pure.
For geometric phases see \cite{Pancharatnam1956,Berry1984,Anandan1992Nature,JohnsFuller2017,JohnsFuller2022};
for empirical summaries see \cite{SuperK1998,SNO2002,KamLAND2003,DayaBay2012,PDG2024Neutrino};
for absolute mass constraints see \cite{KATRIN2025Science,KATRIN2024ArXiv}.

% ==============================================================
\section*{Appendix B: Worked example (constant pitch, BI-bounded)}
With \(\kappa_0=0.80,\ \tau_0=0.90,\ u=\sqrt{\mathcal I}/\eemaxsq=0.50,\ v=\mathcal J/(\eemaxsq\sqrt{\mathcal I})=0.60\), we have
\[
\kappa=\kappa_0 u=0.40,\qquad \tau=\tau_0 v=0.54,\qquad
u^4+(u^2v)^2=0.085<1.
\]
Set the total angle \(\Theta=S\sqrt{\kappa^2+\tau^2}=70.6~\text{rad}\) and an effective \(\Theta_{\rm eff}=\Theta\bmod 2\pi\).

\paragraph{Axis and tilt from \(\kappa,\tau\) (ledger-native).}
Let \(r=\sqrt{\kappa^2+\tau^2}\). Define a small 1--3 skew from non-commutativity,
\[
\zeta \equiv c_\zeta\,v,\qquad v=\frac{\mathcal J}{\eemaxsq\sqrt{\mathcal I}},
\]
and a tilt proportional to the pitch,
\[
\eta \equiv c_\eta\,\frac{2}{\pi}\arctan\!\Big(\frac{\tau}{\kappa}\Big),\qquad 0<c_\zeta,\ c_\eta\lesssim 1.2.
\]
Then
\[
n_{13}^{\rm raw}=\zeta\,\frac{\kappa\tau}{r},\quad
D=\sqrt{r^{2}+(n_{13}^{\rm raw})^{2}},\quad
n_{12}=\frac{\kappa}{D},\ \ n_{23}=\frac{\tau}{D},\ \ n_{13}=\frac{n_{13}^{\rm raw}}{D},
\]
and the antisymmetric/symmetric tilt is
\[
\phi=\eta\,\arctan\!\Big(\frac{\tau}{\kappa}\Big).
\]
This choice is dimensionless and BI-bounded, respects duality parities (even \(\kappa\), odd \(\tau\)), and reduces to the real \(SO(3)\) case (\(J=0\)) when \(v\to 0\) or \(\tau\to 0\).

\paragraph{Two calibrated settings and parameter naturalness.}
The calibration constants \(c_\zeta\) and \(c_\eta\) enter only through the
ledger‐native mappings of the skew and tilt, and represent order‐unity
alignment factors within the BI cap.
They are not fitted parameters in the statistical sense but smooth geometric
scales that remain within the natural range
\[
0 < c_\zeta,\ c_\eta \lesssim 1.2,
\]
ensuring that no fine tuning is required for realistic mixing.

\textit{Tight‐fit (illustrative match).}
For the reference example in this paper we take
\(c_\zeta = 0.92,\ c_\eta = 1.02,\ \Theta_{\mathrm{eff}} = 1.50~\mathrm{rad}\),
values well inside the natural range above.
These yield
\[
\phi \approx 0.65~\mathrm{rad},\qquad
(\vartheta_{12},\vartheta_{23},\vartheta_{13})
  \approx (33.8^\circ,\,48.5^\circ,\,8.7^\circ),
\]
\[
J \simeq -2.2\times10^{-2},\qquad
\delta_{\mathrm{geo}} \approx 232^\circ\ \text{(from }J\text{)}.
\]

\textit{BI‐bounded (first‐principles, untuned).}
Using \(c_\zeta = 1.00,\ c_\eta = 1.00,\ \Theta_{\mathrm{eff}} = 2.10~\mathrm{rad}\)
gives the directly derived, untuned case:
\[
(n_{12},n_{23},n_{13}) \approx (0.581,\,0.784,\,0.219),\quad
\phi \approx 0.61~\mathrm{rad},
\]
\[
(\vartheta_{12},\vartheta_{23},\vartheta_{13})
  \approx (35.6^\circ,\,47.7^\circ,\,7.5^\circ),\qquad
J \approx -2.6\times10^{-2},\qquad
\delta_{\mathrm{geo}} \approx 229^\circ.
\]
Both settings reproduce PDG‐like values within present uncertainties and
demonstrate that the APM/QMU construction achieves realistic neutrino mixing
without external SI parameters or fine‐tuned coefficients.


\paragraph{Eigenphases and mass-ratio check.}
Let the eigenvalues of \(U(H)\) be \(e^{-i\phi_i}\) with \(\phi_i\in(-\pi,\pi]\). These phases define \(\Xi_k=\phi_k\) for BI bounds and feed the ledger ratio law
\[
\frac{m_{\nu_i}}{m_{\nu_j}}=\left|\frac{\phi_i}{\phi_j}\right|^\gamma.
\]
Fix \(\gamma\) by one oscillation scale (e.g.\ \(\Delta\mu_{31}\)) and predict the other. (Alternate trigonometric forms, e.g.\ \(|\sin(\phi_i/2)|\), behave similarly; we retain the phase-ratio form for simplicity.)
\paragraph{Reproducibility.}
A companion script \texttt{apm33\_holonomy\_example.py} reproduces all numbers above (angles, $J$, eigenphases, mass-ratio test) from the ledger inputs \((u,v;\kappa_0,\tau_0,S)\).

% ==============================================================
\section*{Appendix C: Alternative ansätze and robustness}
A parity-respecting power family
\[
\kappa=\kappa_0 u^{p},\qquad \tau=\tau_0 u^{q}v,\qquad p\in[1/3,1],\ q\in[0,2/3],
\]
preserves regularity and BI limits. Varying \(p,q\) within these ranges changes
\((\vartheta_{12},\vartheta_{23},\vartheta_{13})\) by only a few degrees, showing the minimal ansatz of Eq.~\eqref{eq:kappatau} is representative, not fine-tuned.

% ==============================================================
\bibliographystyle{unsrtnat}
\bibliography{qadi_refs}
\end{document}
